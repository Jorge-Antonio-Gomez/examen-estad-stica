\documentclass[8pt]{article}
\usepackage{amsmath}
\usepackage{mathbbol}
\usepackage{amssymb}
\usepackage{amsthm}
\usepackage{amscd}
\usepackage{latexsym}
\usepackage{listings}
\usepackage{color}
\usepackage{hyperref}
\usepackage{blindtext}

\usepackage[a4paper, total={10cm, 26cm}]{geometry}


% Info
\title{Solucionario de Estadística}
\author{Jorge Antonio Gómez García}
\date{\today}

\begin{document}
\maketitle

\tableofcontents

\section{Demostraciones}

    \subsection{Función binomial}
    
    \subsubsection{Función generadora de momentos}

    \textit{D!}
    
    \begin{align*}
        m(t) &:= E[e^{tX}] \\
            &= \sum^n_{k=0}\binom{n}{k}p^kq^{n-k}e^{kt} \\
            &= \sum^n_{k=0}\binom{n}{k}(pe^t)^kq^{n-k} \\
            &= (q+pe^t)^n
    \end{align*}\qed

    Recuerde que $q = (1-p)$:
    
    \begin{align*}
        m(t) = (1-p+pe^t)^n, \,\,\,\, t\in\mathbb{R}
    \end{align*}

    
    
    \subsubsection{Esperanza}
    
    \textit{D!}
    
    \begin{align*}
        E[X] &= \sum^n_{k=0} k\binom{n}{k}p^kq^{n-k} \\
            &= \sum^n_{k=0} \dfrac{n!}{k!(n-k)!} p^kq^{n-k} \\
            &= \sum^n_{k=1}\dfrac{n(n-1)!}{(k-1)!(n-k)!}p^kq^{n-k}  \\
            &= np \sum^n_{k=1} \binom{n-1}{k-1}p^{k-1}q^{n-k} \\
            &= np \sum^{n-1}_{m=0} \binom{n-1}{m} p^{m}q^{n-1-m} \\
            &= np(p+q)^{n-1} \\
            &= np
    \end{align*}\qed

    
    
    \subsubsection{Varianza}

    \begin{flalign*}
        E[X^2] &= \sum^{n}_{k=0} k^2\binom{n}{k}p^kq^{n-k} \\
            &= \sum^{n}_{k=0} \dfrac{n!}{k!(n-k)!} p^kq^{n-k} \\
            &= \sum^{n}_{k=1}\dfrac{n(n-1)!}{(k-1)!(n-k)!}p^kq^{n-k} \\
            &= np \sum^{n}_{k=1} \binom{n-1}{k-1}p^{k-1}q^{n-k} \\
            &= np \sum^{n-1}_{m=0} \binom{n-1}{m} p^{m}q^{n-1-m} \\
            &= np(p+q)^{n-1} \\
            &= np
    \end{flalign*}

    \begin{flalign*}
        Var(X) &:= E[X^2] - (E[X])^2 \\
                &= n(n-1)p^2 + np - (np)^2 \\
                &= npq
    \end{flalign*}

        
\subsection{Función poisson}
        
    \subsubsection{Función generadora de momentos}
    
    \begin{flalign*}
        m(t) &:= E[e^{tX}] \\
            &=   \\
            &=   \\
            &= e^{\lambda(e^t-1)}
    \end{flalign*}

\end{document}
